\section{Ievads}

    Šī projekta galvenais mērķis ir ļaut darba autoriem vairāk izpētīt un iepazīties
    ar .NET un C\# ekosistēmu. Kā arī izveidot tīmekļa aplikāciju izmantojot relatīvi
    jaunu ietvaru Blazor un izveidot ar roku rakstītu ciparu klasifikātoru izmantojot
    MNIST rakstīto ciparu datu kopu.

    Galvenās problēmas ar ko autori saskarsies ir grafisko datu apstrāde Blazor ietvarā,
    jo darba autori izmanto Linux distributīvu, Ubuntu, lai izstrādātu šo projektu, un
    mašīnmācīšanās modeļa izveide, lai tas varētu ar labu precizitāt prognozēt lietotāja
    uzzīmētā cipara skaitli.

    \textbf{Darba mērķi:}

    Izmantojot Blazor framework izveidot tīmekļa aplikāciju, izpētīt pieejamās mašīnmācīšanās
    iespējas C\# un .NET ekosistēmā un izveidot ar roku rakstītu ciparu klasifikātoru.

    \textbf{Darba uzdevumi:}

    \begin{itemize}
        \item Uzstādīt un nokonfigurēt Blazor projektu;
        \item Izpētīt Blazor un C\# grafiskās apstrādes iespējas;
        \item Izveidot tīmekļa aplikāciju izmantojot Blazor;
        \item Nodrošināt lietotāja ievadīto datu apstrādi;
        \item Izpētīt .NET un C\# mašīnmācīšanās iespējas;
        \item Izvēlēties ietvaru vai bibliotēku priekš mašīnmācīšanās modeļa izstrādes;
        \item Iegūt MNIST datus un izveidot parseri, kas ielasa datus atmiņā un noparsē šos datus;
        \item Izveidot modeli un programmu, kas apmāca šo modeli;
        \item Savilkt kopā tīmekļa aplikāciju un mašīnmācīšanās modeli nodrošinot, ka modelim tiek padoti pareizi dati;
    \end{itemize}

