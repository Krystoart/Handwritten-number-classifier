\section{Ievads}

    Šī projekta galvenais mērķis ir izpētīt un iepazīties
    ar \texttt{.NET} un \texttt{C\#} ekosistēmu. Projekta uzdevums ir izveidot tīmekļa aplikāciju, izmantojot relatīvi
    jaunu ietvaru \texttt{Blazor}, un izveidot klasifikatoru ar roku rakstītiem cipariem, izmantojot
    \texttt{MNIST} \cite{MNISTHandwrittenDigit} rakstīto ciparu datu kopu.

    \textbf{Darba mērķi:}

    Izmantojot \texttt{Blazor framework}, izveidot tīmekļa aplikāciju, izpētīt pieejamās mašīnmācīšanās
    iespējas \texttt{C\#} un \texttt{.NET} ekosistēmā un izveidot klasifikatoru ar roku rakstītiem cipariem.

    \textbf{Darba uzdevumi:}

    \begin{itemize}
        \item Uzstādīt un nokonfigurēt \texttt{Blazor} projektu;
        \item Izpētīt \texttt{Blazor} un \texttt{C\#} grafiskās apstrādes iespējas;
        \item Izveidot tīmekļa aplikāciju, izmantojot \texttt{Blazor};
        \item Nodrošināt lietotāja ievadīto datu apstrādi;
        \item Izpētīt .\texttt{NET} un \texttt{C\#} mašīnmācīšanās iespējas;
        \item Izvēlēties ietvaru vai bibliotēku priekš mašīnmācīšanās modeļa izstrādes;
        \item Iegūt \texttt{MNIST} datus un izveidot parsētāju, kas ielasa datus atmiņā un parsē tos;
        \item Izveidot modeli un programmu, kas apmāca šo modeli;
        \item Apvienot tīmekļa aplikāciju un mašīnmācīšanās modeli, nodrošinot, ka modelim tiek padoti pareizi dati;
    \end{itemize}

