\section{Projekta apraksts}

    Tika izveidota tīmekļa aplikācija, izmantojot \texttt{Blazor} ietvaru, kur lietotājs
    ar pelītes palīdzību var uzzīmēt, piemēram, kādu ciparu, un tika izveidots mašīnmācīšanās modelis,
    izmantojot \texttt{MNIST} datu kopu, kas analizē uzzīmēto ciparu un atgriež procentuālu sadalījumu ar datu kopas klasēm. Klase ar augstāko rezultātu paredz uzzīmēto ciparu. Tā kā projekts ir apjomīgs, tas tika veikts grupā. Projekts tika
    sadalīts divās daļās - tīmekļa aplikācijas izstrāde un mašīnmācīšanās modeļa izstrāde. Attiecīgi autori
    izvēlējās darbu sadalīt:

    \begin{itemize}
        \item Arina Solovjova atbildīga par uzdevumiem saistībā ar tīmekļa
    aplikācijas izstrādi
        \item Kristofers Volkovs atbildīgs par mašīnmācīšanās modeļa izstrādi.
    \end{itemize}

    % Description about the created web page
    \subsection{Tīmekļa aplikācija}
Tīmekļa aplikācijas lapas sastāvā galvenie elementi ir zīmēšanas \texttt{canvas}, uz kura lietotājs var zīmēt izvēlēto ciparu. Tas sastāv no \texttt{svg} elementa, kurš uztur vairākus \texttt{polyline} elementus. Elementam ir piesaistītas tādas funkcijas, kā zīmējuma pilnīga izdzēšana, pēdējās darbības atcelšana, iespēja lejupielādēt uzzīmēto attēlu un iespēja padot zīmējumu mašīnmācīšanās algoritmam, lai tas analizētu uzzīmēto.
\par Ir papildus pievienota zīmējumu un minējumu vēsture. Zīmējumi tiek attēloti samazinātā formātā ar mašīnmācīšanās algoritma rezultātiem. Zīmējumu vēstures elementus ir iespējams izdzēst, atsevišķi atlasot noteikto zīmējumu un uzspiežot uz pogas 'DELETE'.





    % Description about the created ML model
    \subsection{Mašīnmācīšanās modelis}

    Mašīnmācīšanās modeļa izstrādi ir iespējams sadalīt vēl divās daļās: modeļa trennēšana un modeļa
    implementācija tīmekļa aplikācijā.

    Sākotnēji tika veikta modeļa trennēšana, bet vēl pirms tā tika veikta izpēte par to kādi gatavi
    ietvari jau eksistē \texttt{C\#} ekosistēmā. Pēc vairāku ietvaru izpētes kā \texttt{PyTorch.NET},
    \texttt{Keras.NET}, \texttt{ML.NET} un \texttt{CNTK}. Sākumā bija plānots izmantot \texttt{CNTK},
    bet izpētes processā tika secināts, ka Microsoft ir pārtraukuši atbalstu \texttt{CNTK} un vairs
    to neuzlabos un neatjaunos, tapēc beigas tika izvēlēts \texttt{ML.NET}.

    Attiecīgi tika izveidota programma, kas apmāca modeli izmantojot \texttt{MNIST} datu kopu un
    šīs programmas darbība ir aprakstīta \ref{ml:train}~attēlā.

    \begin{figure}[H]
        \centering
        \includegraphics[width=15cm]{VPL-ML.png}
        \caption{ML modeļa trennēšanas diagramma}
        \label{ml:train}
    \end{figure}

