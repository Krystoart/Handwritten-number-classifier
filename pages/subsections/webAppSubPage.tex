\subsection{Tīmekļa aplikācija}
Tīmekļa aplikācijas lapas sastāvā galvenie elementi ir zīmēšanas \texttt{canvas}, uz kura lietotājs var zīmēt izvēlēto ciparu. Tas sastāv no \texttt{svg} elementa, kurš uztur vairākus \texttt{polyline} elementus. Elementam ir piesaistītas tādas funkcijas, kā zīmējuma pilnīga izdzēšana, pēdējās darbības atcelšana, iespēja lejupielādēt uzzīmēto attēlu un iespēja padot zīmējumu mašīnmācīšanās algoritmam, lai tas analizētu uzzīmēto.
\par Zīmējuma izdzēšanas funkcija ir pievienota pie pogas 'CLEAR'. Funkcija izdzēš visus uz zīmēšanas \texttt{canvas} uzzīmētos \texttt{polyline} elementus, kas ir saglabāti. Funkcija, kas tiek izsaukta ar pogu 'UNDO', izdzēš pēdējo uz zīmēšanas \texttt{canvas} uzzīmēto \texttt{polyline} elementu. Poga 'GUESS' izsauc funkciju, kas padod uzzīmēto ciparu mašīnmācīšanās algoritmam. Poga'SAVE' izsauc funkciju, kas saglabā uzzīmēto lietotāja datorā.
\par Ir papildus pievienota zīmējumu un minējumu vēsture. Zīmējumi tiek attēloti samazinātā formātā ar mašīnmācīšanās algoritma rezultātiem. Zīmējumu vēstures elementus ir iespējams izdzēst, atsevišķi atlasot noteikto zīmējumu un uzspiežot uz pogas 'DELETE'. Poga izsauc funkciju, kas izdzēš noteikto elementu no saglabāto zīmējumu saraksta.
