\subsection{Mašīnmācīšanās modelis}

    Šajā sadaļā tiks apskatītas tikai galvenās koda daļas, gan priekš
    modeļa apmācīšanas, gan modeļa implementācijas.

    Trenēšanas procesā bija nepieciešams \texttt{MNIST} datus sakārtotā formā uzglabāt atmiņā, tāpēc
    tika izveidotas 2 klases: \texttt{MnistList} un \texttt{MnistItem}. \texttt{MnistItem} klase tieši
    uzglabā pikseļu vērtības un attiecīgo bildes ciparu. \texttt{MnistList} ir klases mainīgais, kurā
    tiek uzglabāts masīvs ar \texttt{MnistItem} klasēm. \texttt{MnistItem} ir klase, kurā tiek uzglabāta
    viena bilde un šīs bildes cipars. Šai klasei ir 2 galvenie mainīgie: \texttt{Pixels}, kas ir masīvs
    ar float vērtībām, un \texttt{Label}, kas ir float vērtība. Lai ielasītu datus, ir izveidota parsēšanas
    funkcija, kas ir \texttt{MnistList} konstruktors, un to ir iespējams apskatīt sekojoši:

    \begin{minted}{csharp}
    public MnistList(string imagePath, string labelPath)
    {
        FileStream fsImages = new FileStream(
            imagePath, FileMode.Open
        ); // Images
        FileStream fsLabels = new FileStream(
            labelPath, FileMode.Open
        ); // Labels

        BinaryReader brImages = new BinaryReader(fsImages);
        BinaryReader brLabels = new BinaryReader(fsLabels);

        int magic1 = ReverseBytes(brImages.ReadInt32());
        int magic2 = ReverseBytes(brLabels.ReadInt32());

        // Tests if dataset magic numbers are correct
        if (magic1 != 2051)
            throw new Exception("Not a valid MNIST image data set");

        if (magic2 != 2049)
            throw new Exception("Not a valid MNIST label data set");

        int imgCount = ReverseBytes(brImages.ReadInt32());
        int labelCount = ReverseBytes(brLabels.ReadInt32());

        // Checks if for each image there is a label
        if (imgCount != labelCount)
            throw new Exception(
                "Number of items of the two files is not the same"
            );

        int imgRows = ReverseBytes(brImages.ReadInt32());
        int imgCols = ReverseBytes(brImages.ReadInt32());

        this.Length = imgCount;
        this.Rows = imgRows;
        this.Columns = imgCols;

        this.images = new MnistItem[this.Length];
        byte[][] item = new byte[Rows][];
        for (int i = 0; i < item.Length; i++)
            item[i] = new byte[Columns];

        for (int di = 0; di < this.Length; ++di)
        {
            for (int i = 0; i < item.Length; ++i)
            {
                for (int j = 0; j < item.Length; j++)
                {
                    byte b = brImages.ReadByte();
                    item[i][j] = b;
                }
            }
            byte label = brLabels.ReadByte();

            MnistItem newImg = new MnistItem(
                width: 28, height: 28, pixels: item, label: label
            );
            images[di] = newImg;
        }

        fsImages.Close();
        brImages.Close();
        fsLabels.Close();
        brLabels.Close();
    }
\end{minted}

    Par piemēru šīm klasēm tika ņemts \texttt{GitHub} piemērs \cite{paxbunPaxbunCntkMnistPractice2019}.
    \texttt{MNIST} dati ir binārie dati un to pikseļu vērtības ir intervālā no 0 līdz 255,
    bet, ja modelim tiks doti normalizēti dati, kas ir vērtībās no 0 līdz 1, tad tas
    uztrenēsies labāk. Tāpēc \texttt{MnistItem} klases konstruktorā šie dati tiek
    normalizēti.

    \begin{minted}{csharp}
    public MnistItem(int width, int height, byte[][] pixels, byte label)
    {
        this.Width = width;
        this.Height = height;
        this.Label = label;

        this.Pixels = new float[height*width];

        int counter = 0;
        for (int i = 0; i < height; ++i)
        {
            for (int j = 0; j < width; ++j)
            {
                this.Pixels[counter] = ((float)pixels[i][j]) / byte.MaxValue;
                counter += 1;
            }
        }
    }
\end{minted}

    Pēc datu ielasīšanas ir nepieciešams apmācīt modeli. Priekš tā tika izveidota funkcija, tika ņemts jau eksistējošs piemērs no \texttt{ML.NET} \texttt{GitHub}
    repozitorija \cite{DotnetMachinelearningsamples2021}. Dotajā piemērā izmantotā datu kopa
    neatbilda īstajai \texttt{MNIST} datu kopai, bet gan daudz mazākai datu kopai, tāpēc šo piemēru bija
    jāpārveido, lai tas darbotos ar nepieciešamo \texttt{MNIST} datu kopu. \cite{MNISTHandwrittenDigit}

    \begin{minted}{csharp}
    public static void Train(
        MLContext mlContext, MnistList trainingData, MnistList testingData
    )
    {
        try
        {
            // 1: Data loaded into mlContext
            var trainData = mlContext.Data.LoadFromEnumerable<MnistItem>(
                trainingData.images
            );
            var testData = mlContext.Data.LoadFromEnumerable<MnistItem>(
                testingData.images
            );

            // 2: Context data process configuration with pipeline
            // data transformations
            var dataProcessPipeline = mlContext.Transforms.
                Conversion.MapValueToKey(
                    "Label", "Label",
                    keyOrdinality: ValueToKeyMappingEstimator.
                        KeyOrdinality.ByValue
                ).
                Append(
                    mlContext.Transforms.Concatenate(
                        "Features", nameof(MnistItem.Pixels)
                    ).AppendCacheCheckpoint(mlContext)
                );

            // 3: Set the training algorithm, then create and config
            // the modelBuilder
            var trainer = mlContext.MulticlassClassification.
                Trainers.SdcaMaximumEntropy(
                    labelColumnName: "Label", featureColumnName: "Features"
                );
            var trainingPipeline = dataProcessPipeline.Append(trainer);

            // 4: Train the model fitting to the DataSet
            Console.WriteLine(
                "=============== Training the model ==============="
            );
            ITransformer trainedModel = trainingPipeline.Fit(trainData);

            Console.WriteLine(
                "===== Evaluating Model's accuracy with Test data ====="
            );
            var predictions = trainedModel.Transform(testData);
            var metrics = mlContext.
                MulticlassClassification.Evaluate(
                    data: predictions, labelColumnName: "Label",
                    scoreColumnName: "Score"
                );

            Common.ConsoleHelper.PrintMultiClassClassificationMetrics(
                trainer.ToString(), metrics
            );

            // If there is already a trained model then this will
            // override that model
            mlContext.Model.Save(trainedModel, trainData.Schema, ModelPath);

            Console.WriteLine("The model is saved to {0}", ModelPath);
        }
        catch (Exception ex)
        {
            Console.WriteLine(ex.ToString());
        }
    }
\end{minted}

    Rezultātā tika izveidots modelis, kuru bija nepieciešams implementēt tīmekļa aplikācijā.
    Tika izveidota klase \texttt{MnistClassificator} ar metodi, kura saņem masīvu ar pikseļu vērtībām un
     atgriež masīvu ar float vērtībām, kur katrs elements
    atbilst kādai no ciparu klasēm.

    \begin{minted}{csharp}
    /*
    Input parameters
    ----------------
    image: 784 array, each value is between 0 and 255 (white - black)
    */
    public float[] Analyze(byte[] image)
    {
        if (image.Length != 784)
            throw new Exception("Image length is not 784.");

        MnistItem imageObject = new MnistItem(
            length: image.Length, pixels: image
        );

        MLContext mlContext = new MLContext();
        var predEngine = mlContext.Model.
            CreatePredictionEngine<MnistItem, MnistOutPutData>(
                this.trainedModel
            );

        var output = predEngine.Predict(imageObject);

        return output.Score;
    }
\end{minted}

    Lai ielasītu modeli atmiņā un to varētu izmantot tiek izmantota \texttt{ML.NET} funkcija:
    \mintinline{csharp}{this.trainedModel = mlContext.Model.Load(ModelPath, out var modelInputSchema);}
