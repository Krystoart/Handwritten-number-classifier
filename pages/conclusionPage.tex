\section{Secinājumi}

    Vispārīgi darba koncepts ir samērā vienkāršs, un noteikti cilvēki, kas ar šādām tehnoloģijām būtu jau
    ilgāku laiku strādājuši, šāda veida projekta izveide būtu vienkāršs process, bet tā kā darba autoriem
    šī bija pirmā pieredze strādājot ar šādām tehnoloģijām, tad šī projekta izveide sagādāja dažādas
    grūtības. Projekta izveides processā abi darba autori daudz ko jaunu iemācījās un izsecināja
    dažādas lietas par C\# ekosistēmu.

    Galvenie secinājumi:
    \begin{itemize}
        \item C\# valodai un ietvariem ir labs atbalsts uz Windows operētājsistēmas;
        \item Strādāt ar C\# ietvariem un bibliotēkām ārpus Windows operētājsistēmas ir ļoti apgrūtinoši;
        \item Vispārīgi C\# mašīnmācīšanās ietvariem ir ļoti slikts uz unix bāzētām operētājsistēmām;
        \item C\# mašīnmācīšanās ietvari nenodrošina labu "\textit{user experience}";
        \item Liela daļa C\# mašīnmācīšanās ietvari ir vienkārši "\textit{api wrappers}" pāri python ietvariem kā, piemēram, PyTorch un TensorFlow;
        \item ML.NET ir samērā spēcīgs rīks priekš mazu mašīnmācīšanās modeļu izveides;
        \item ML.NET ir tik ļoti abstrahēts, ka to ir ļoti grūti izmantot un izprast;
        \item Ir ļoti svarīgi izveidot pareizu modeļa arhitektūru, lai tas strādātu pareizi arī nestandarta apstākļos;
        \item Python un tā ietvari piedāvā labāku atbalstu priekš mašīnmācīšanās modeļu izveides;
    \end{itemize}
