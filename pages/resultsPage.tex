\section{Rezultāti}

Izstrādātās tīmekļa aplikācijas galvenie elementi ir redzami \ref{orig:lapasElementi} attēlā. Attēla kreisajā pusē ir redzams \texttt{svg} elements, uz kura lietotājs var zīmēt savu izvēlēto ciparu. Zem šī elementa ir pogas ar papildus funkcijām (\ref{orig:svgElementi} attēls). Uzspiežot pogu 'GUESS', zīmējums tiek saglabāts, analizēts un rezultāts ar zīmējuma samazināto versiju ir redzams attēla labajā pusē.

\begin{figure}[H]
    \centering
    \fbox{\includegraphics[width=120mm]{history.jpeg}}
    \caption{Tīmekļa aplikācijas elementi}
    \label{orig:lapasElementi}
\end{figure}

\begin{figure}[H]
    \centering
    \fbox{\includegraphics[width=120mm]{drawing.jpeg}}
    \caption{Galvenais svg elements un pogas}
    \label{orig:svgElementi}
\end{figure}

\par Zīmējumu vēstures skats ir redzams \ref{orig:selection} attēla labajā pusē. Sarakstam ir pielietota iespēja izvēlēties kādu no elementiem un izdzēst to no vēstures.

\begin{figure}[H]
    \centering
    \fbox{\includegraphics[width=120mm]{selection.jpeg}}
    \caption{Zīmējumu vēsture}
    \label{orig:selection}
\end{figure}

\par Viena no zīmējumam pievienotajām funckijām ir attēla saglabāšana. Šī funkcija nav izmantota nekur citur projektā. Tā nenes nekādu citu jēgu kā ļaut lietotājam lejupielādēt savu zīmējumu. Tā tiek pielietota, lietotājam nospiežot 'SAVE' pogu, kas redzama \ref{orig:svgElementi} attēlā. Ar šo pogu zīmējums tiek saglabāts kā png fails lietotāja datorā ar nosaukumu \textit{number.png}. \ref{orig:downloadedFile} attēlā redzams lejupielādētais zīmējums.
\begin{figure}[H]
    \centering
    \fbox{\includegraphics[width=120mm]{downloaded.jpeg}}
    \caption{Lejupielādētais attēls}
    \label{orig:downloadedFile}
\end{figure}

